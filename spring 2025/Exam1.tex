% Options for packages loaded elsewhere
\PassOptionsToPackage{unicode}{hyperref}
\PassOptionsToPackage{hyphens}{url}
\PassOptionsToPackage{dvipsnames,svgnames,x11names}{xcolor}
%
\documentclass[
  letterpaper,
  DIV=11,
  numbers=noendperiod]{scrartcl}

\usepackage{amsmath,amssymb}
\usepackage{iftex}
\ifPDFTeX
  \usepackage[T1]{fontenc}
  \usepackage[utf8]{inputenc}
  \usepackage{textcomp} % provide euro and other symbols
\else % if luatex or xetex
  \usepackage{unicode-math}
  \defaultfontfeatures{Scale=MatchLowercase}
  \defaultfontfeatures[\rmfamily]{Ligatures=TeX,Scale=1}
\fi
\usepackage{lmodern}
\ifPDFTeX\else  
    % xetex/luatex font selection
\fi
% Use upquote if available, for straight quotes in verbatim environments
\IfFileExists{upquote.sty}{\usepackage{upquote}}{}
\IfFileExists{microtype.sty}{% use microtype if available
  \usepackage[]{microtype}
  \UseMicrotypeSet[protrusion]{basicmath} % disable protrusion for tt fonts
}{}
\makeatletter
\@ifundefined{KOMAClassName}{% if non-KOMA class
  \IfFileExists{parskip.sty}{%
    \usepackage{parskip}
  }{% else
    \setlength{\parindent}{0pt}
    \setlength{\parskip}{6pt plus 2pt minus 1pt}}
}{% if KOMA class
  \KOMAoptions{parskip=half}}
\makeatother
\usepackage{xcolor}
\setlength{\emergencystretch}{3em} % prevent overfull lines
\setcounter{secnumdepth}{-\maxdimen} % remove section numbering
% Make \paragraph and \subparagraph free-standing
\ifx\paragraph\undefined\else
  \let\oldparagraph\paragraph
  \renewcommand{\paragraph}[1]{\oldparagraph{#1}\mbox{}}
\fi
\ifx\subparagraph\undefined\else
  \let\oldsubparagraph\subparagraph
  \renewcommand{\subparagraph}[1]{\oldsubparagraph{#1}\mbox{}}
\fi


\providecommand{\tightlist}{%
  \setlength{\itemsep}{0pt}\setlength{\parskip}{0pt}}\usepackage{longtable,booktabs,array}
\usepackage{calc} % for calculating minipage widths
% Correct order of tables after \paragraph or \subparagraph
\usepackage{etoolbox}
\makeatletter
\patchcmd\longtable{\par}{\if@noskipsec\mbox{}\fi\par}{}{}
\makeatother
% Allow footnotes in longtable head/foot
\IfFileExists{footnotehyper.sty}{\usepackage{footnotehyper}}{\usepackage{footnote}}
\makesavenoteenv{longtable}
\usepackage{graphicx}
\makeatletter
\def\maxwidth{\ifdim\Gin@nat@width>\linewidth\linewidth\else\Gin@nat@width\fi}
\def\maxheight{\ifdim\Gin@nat@height>\textheight\textheight\else\Gin@nat@height\fi}
\makeatother
% Scale images if necessary, so that they will not overflow the page
% margins by default, and it is still possible to overwrite the defaults
% using explicit options in \includegraphics[width, height, ...]{}
\setkeys{Gin}{width=\maxwidth,height=\maxheight,keepaspectratio}
% Set default figure placement to htbp
\makeatletter
\def\fps@figure{htbp}
\makeatother
% definitions for citeproc citations
\NewDocumentCommand\citeproctext{}{}
\NewDocumentCommand\citeproc{mm}{%
  \begingroup\def\citeproctext{#2}\cite{#1}\endgroup}
\makeatletter
 % allow citations to break across lines
 \let\@cite@ofmt\@firstofone
 % avoid brackets around text for \cite:
 \def\@biblabel#1{}
 \def\@cite#1#2{{#1\if@tempswa , #2\fi}}
\makeatother
\newlength{\cslhangindent}
\setlength{\cslhangindent}{1.5em}
\newlength{\csllabelwidth}
\setlength{\csllabelwidth}{3em}
\newenvironment{CSLReferences}[2] % #1 hanging-indent, #2 entry-spacing
 {\begin{list}{}{%
  \setlength{\itemindent}{0pt}
  \setlength{\leftmargin}{0pt}
  \setlength{\parsep}{0pt}
  % turn on hanging indent if param 1 is 1
  \ifodd #1
   \setlength{\leftmargin}{\cslhangindent}
   \setlength{\itemindent}{-1\cslhangindent}
  \fi
  % set entry spacing
  \setlength{\itemsep}{#2\baselineskip}}}
 {\end{list}}
\usepackage{calc}
\newcommand{\CSLBlock}[1]{\hfill\break\parbox[t]{\linewidth}{\strut\ignorespaces#1\strut}}
\newcommand{\CSLLeftMargin}[1]{\parbox[t]{\csllabelwidth}{\strut#1\strut}}
\newcommand{\CSLRightInline}[1]{\parbox[t]{\linewidth - \csllabelwidth}{\strut#1\strut}}
\newcommand{\CSLIndent}[1]{\hspace{\cslhangindent}#1}

\KOMAoption{captions}{tableheading}
\makeatletter
\@ifpackageloaded{caption}{}{\usepackage{caption}}
\AtBeginDocument{%
\ifdefined\contentsname
  \renewcommand*\contentsname{Table of contents}
\else
  \newcommand\contentsname{Table of contents}
\fi
\ifdefined\listfigurename
  \renewcommand*\listfigurename{List of Figures}
\else
  \newcommand\listfigurename{List of Figures}
\fi
\ifdefined\listtablename
  \renewcommand*\listtablename{List of Tables}
\else
  \newcommand\listtablename{List of Tables}
\fi
\ifdefined\figurename
  \renewcommand*\figurename{Figure}
\else
  \newcommand\figurename{Figure}
\fi
\ifdefined\tablename
  \renewcommand*\tablename{Table}
\else
  \newcommand\tablename{Table}
\fi
}
\@ifpackageloaded{float}{}{\usepackage{float}}
\floatstyle{ruled}
\@ifundefined{c@chapter}{\newfloat{codelisting}{h}{lop}}{\newfloat{codelisting}{h}{lop}[chapter]}
\floatname{codelisting}{Listing}
\newcommand*\listoflistings{\listof{codelisting}{List of Listings}}
\makeatother
\makeatletter
\makeatother
\makeatletter
\@ifpackageloaded{caption}{}{\usepackage{caption}}
\@ifpackageloaded{subcaption}{}{\usepackage{subcaption}}
\makeatother
\ifLuaTeX
  \usepackage{selnolig}  % disable illegal ligatures
\fi
\usepackage{bookmark}

\IfFileExists{xurl.sty}{\usepackage{xurl}}{} % add URL line breaks if available
\urlstyle{same} % disable monospaced font for URLs
\hypersetup{
  pdftitle={The Effects of Neuromuscular Activity and Muscle Structure on Stepping Performance in Older Adults},
  pdfauthor={Ann Kim, Michael Peña, Julie Werner},
  colorlinks=true,
  linkcolor={blue},
  filecolor={Maroon},
  citecolor={Blue},
  urlcolor={Blue},
  pdfcreator={LaTeX via pandoc}}

\title{The Effects of Neuromuscular Activity and Muscle Structure on
Stepping Performance in Older Adults}
\usepackage{etoolbox}
\makeatletter
\providecommand{\subtitle}[1]{% add subtitle to \maketitle
  \apptocmd{\@title}{\par {\large #1 \par}}{}{}
}
\makeatother
\subtitle{Math 539 - Exam 1}
\author{Ann Kim, Michael Peña, Julie Werner}
\date{2025-02-19}

\begin{document}
\maketitle
\begin{abstract}
\noindent Report was prepared as part of Math 539 Statistical Consulting
class. Client for this project is Dr.~Marcel Lanza of the University of
Maryland.
\end{abstract}

\renewcommand*\contentsname{Table of contents}
{
\hypersetup{linkcolor=}
\setcounter{tocdepth}{3}
\tableofcontents
}
\newpage

\section{Introduction}\label{introduction}

Falls pose a significant threat to older adults, with nearly one in four
individuals aged 65 and older experiencing a fall each year that often
results in severe injuries and a loss of independence. In response our
topic is The Effects of Neuromuscular Activity and Muscle Structure on
Stepping Performance in Older Adults''. This study examines age-related
differences in stepping performance and muscular structure: a critical
component of balance recovery by evaluating factors such as stepping
speed, weight transfer speed, and neuromuscular activation, while also
investigating how muscle structure including muscle size, stiffness, and
intramuscular fat content influences movement ability in both younger
and older populations. The research integrates detailed assessments from
electromyography and ultrasound imaging with clinical evaluations
including the Stair Climb Power Test, Timed Up-and-Go Test, Four Square
Step Test, and Handgrip Strength Test to control for confounding factors
and provide comprehensive insights. Ultimately, these findings aim to
inform targeted fall prevention strategies and mobility interventions
designed to enhance stability and maintain independence among older
adults.

\section{Goals and Objectives}\label{goals-and-objectives}

\subsection{Goals}\label{goals}

This study aims to examine age-related differences in stepping
performance and investigate how muscle structure influences movement
ability in younger and older adults. Specifically, it analyzes stepping
speed, weight transfer speed, and neuromuscular activity to identify
potential differences between age groups. By exploring these factors,
this research seeks to provide valuable insights that can inform fall
prevention strategies and mobility interventions for older adults.

\subsection{Objectives}\label{objectives}

To achieve these goals, the study will measure stepping speed and weight
transfer speed in both younger and older adults. It will also analyze
lateral, forward, and backward stepping performance to detect potential
age-related impairments. Additionally, the research will assess the
relationship between muscle structure---specifically muscle size,
stiffness, and fat content---and stepping ability. By comparing stepping
performance factors across age groups, the study aims to uncover
neuromuscular differences that may contribute to balance and mobility
challenges in aging populations.

Ultimately, the findings from this research will enhance understanding
of how aging affects stepping performance and how muscle composition and
function influence movement ability. This knowledge may guide the
development of targeted fall prevention strategies and mobility
interventions to support older adults in maintaining stability and
independence.

\section{Literature Review}\label{literature-review}

\subsection{Scope of the Problem}\label{scope-of-the-problem}

Older adults are at a significantly higher risk of falling compared to
younger adults, with approximately one in four adults aged 65 and older
experiencing a fall each year (National Council on Aging 2024).
Age-related declines in sensory perception, motor control, and cognitive
processing contribute to impaired balance, increasing susceptibility to
falls (National Safety Council 2024). The vestibular, visual, and
proprioceptive systems, which are critical for maintaining postural
stability, deteriorate with age, leading to slower reaction times and
reduced coordination (Fitzpatrick and McCloskey 1994; Kandel et al.
2021). As a result, older adults struggle to recover from sudden balance
disturbances. In addition, muscle function and strength decline with
age. Together, these factors make older adults more prone to falls.

The consequences of falls are severe, as they are the leading cause of
injury-related deaths among older adults (Centers for Disease Control
and Prevention 2024b). Falls frequently result in hip fractures, head
trauma, and other serious injuries, leading to long-term disability,
loss of independence, and increased healthcare costs (National Safety
Council 2024). In 2022 alone, over 3.5 million older adults required
emergency medical attention due to falls (Centers for Disease Control
and Prevention 2024a). These incidents contribute to a substantial
burden on the healthcare system and significantly impact the quality of
life for aging individuals. Given the high prevalence and serious
consequences of falls, fall prevention strategies are essential to
reduce injury risks and improve overall well-being in older adults.

\subsection{Physiology of Balance
Recovery}\label{physiology-of-balance-recovery}

One key component of fall prevention is understanding balance
recovery---the process in which a person who loses balance avoids
falling by using strategies like stepping. The physiology of balance
recovery involves the coordinated interaction of the brain, nerves, and
muscles to maintain or restore postural stability after a disturbance in
balance. This relies on sensory input to detect imbalance and provide
real-time updates, central processing to integrate sensory information
and initiate action, and motor output, which is the execution of a
balance recovery response.

\subsubsection{Sensory Input}\label{sensory-input}

There are three primary sensory systems that the body uses to detect
sway: vestibular, visual, and proprioceptive (Fitzpatrick and McCloskey
1994). The vestibular system detects rotational movements, acceleration,
and head position relative to gravity, and transmits this information
from the vestibular labyrinth in the inner ear {[}Figure 1A{]} via the
vestibular nerve to the vestibular nuclei in the brainstem {[}Figure
1B{]} (Purves et al. 2001). The eyes detect body orientation relative to
the environment, and this visual information is processed in the
occipital cortex of the brain and integrated with other sensory
information about body position and orientation in the brain's parietal
lobe {[}Figure 2{]}(Kandel et al. 2021). Proprioceptive information is
first detected by special sensory receptors in the body that detect
physical forces like pressure, stretch, and vibration, called
mechanoreceptors. Different types of mechanoreceptors are located in the
muscle spindles, Golgi tendon organs, and joint receptors. Muscle
spindles detect how fast a muscle is stretching, Golgi tendon organs
detect muscle tension, and joint receptors detect movement, pressure,
and position of the joints {[}Figure 3{]}(Kandel et al. 2021). These
signals provide feedback about body position and limb movement conveyed
via peripheral nerves to the spinal cord, brainstem, and primary sensory
cortex {[}Figure 4{]} (Kandel et al. 2021).

\begin{figure}[H]

{\centering \includegraphics{/Figures_for_report_PNG/Fig1.png}

}

\caption{(A) Movements of the head are detected by the vestibular
labyrinth structures in the inner ear, (B) The information is then
transmitted via the vestibulocochlear nerve to the vestibular nuclei and
reticular formation in the brainstem.}

\end{figure}%

\footnote{Anatomical figures adapted from Marieb and Wilhelm (2020)}

\begin{figure}[H]

{\centering \includegraphics{/Figures_for_report_PNG/Fig2.png}

}

\caption{Visual information about balance is detected by the eyes and
transmitted to the visual cortex in the occipital lobe of the brain.}

\end{figure}%%
\begin{figure}[H]

{\centering \includegraphics{/Figures_for_report_PNG/Fig3.png}

}

\caption{Two types of sensory receptors that contribute to
proprioception are muscle spindles and Golgi tendon organs. Joint
receptors (not pictured) are a third type.}

\end{figure}%%
\begin{figure}[H]

{\centering \includegraphics{/Figures_for_report_PNG/Fig4.png}

}

\caption{Information about body position and limb movement is
transmitted up the spinal cord, brainstem, and into the primary sensory
cortex in the parietal lobe of the brain. It is integrated with visual
information from the occipital lobe, vestibular information from the
brainstem, and sent to the primary motor cortex in the frontal lobe of
the cerebral cortex.}

\end{figure}%

\subsubsection{Central Processing}\label{central-processing}

After receiving sensory information from the body, the brainstem,
cerebellum, and cerebral cortex integrate this information and transmit
motor commands (Kandel et al. 2021). The vestibular nuclei and reticular
formation in the brainstem integrate vestibular, visual, and
proprioceptive inputs to form motor commands for quick postural
adjustments. The cerebellum is involved with fine-tuning balance by
adjusting muscle activity based on sensory information and plays a role
in motor learning and adaptation over time (Brooks et al. 2015).
Conscious balance strategies and voluntary postural corrections---such
as stepping to recover from a stumble---are produced mainly in the
premotor and primary motor cortices of the frontal lobe, as well as some
processing in the parietal lobe of the cerebral cortex {[}Figure 4{]}.

\subsubsection{Motor Output}\label{motor-output}

Motor output in balance recovery is the execution of an observable
response. This is accomplished in coordination with eye movement, such
as the vestibulo-ocular reflex and several spinal cord reflexes, which
provide immediate corrections to maintain balance and posture (Robinson
2022; Bonsu et al. 2021). For example, a sudden stretch of the hip
flexors---muscles on the front of the hip joint---during unexpected
backward sway would result in hip flexor muscle activation in an attempt
to recover balance. In addition to immediate corrections brought about
by spinal cord reflexes, more complex muscle activation patterns occur,
such as in the ankle, where muscles responsible for lifting the foot,
called dorsiflexors (e.g., tibialis anterior), or muscles responsible
for pushing the foot downward, called plantarflexors (e.g.,
gastrocnemius, soleus) {[}Figure 5A{]}, trigger small ankle movements to
restore equilibrium. Large disturbances in balance result in a
coordinated step to the side, forward, or backward to prevent falling.
Of particular interest to the current investigation is hip muscle
activation and stepping strategy. Stepping necessarily relies on hip
muscle activation strategy. Hip strategy may engage the hip flexor
muscles such as iliopsoas during forward stepping, hip abductor muscles
(used for moving the lower limb out to the side of the body) such as
gluteus medius during lateral stepping {[}Figure 5B, 5C{]}, and hip
extensors (used for moving the lower limb to the rear of the body) such
as gluteus maximus during backward stepping {[}Figure 5B, 5C{]}.

\begin{figure}[H]

{\centering \includegraphics{/Figures_for_report_PNG/Fig5.png}

}

\caption{(A) Tibialis anterior lifts the foot while gastrocnemius and
soleus press the foot down, (B) Gluteus medius moves the lower limb out
to the side of the body at the hip joint while gluteus maximum moves it
to the rear of the body, (C) Iliopsoas is two conjoint muscles which
move the lower limb toward the front of the body, such as during forward
stepping.}

\end{figure}%

\subsection{Features of Muscles Related to Balance
Recovery}\label{features-of-muscles-related-to-balance-recovery}

The speed and quality of motor output when executing a balance recovery
stepping response is dependent not only on sensory input, central
processing, and motor signals from the nervous system, but also on the
quality of muscle function, called neuromuscular activation, and muscle
structure characteristics. Neuromuscular activity refers to the
activation of muscles through neural signals, which can be represented
by electromyogram (EMG) amplitude---a measure of the electrical activity
of muscles fibers, indicating the level of muscle activation and force
generation (Kamen and Caldwell 1996). Muscle morphology (structural)
characteristics of interest are muscle size, muscle quality, muscle
stiffness, and intramuscular fat. Muscle size represents muscle
thickness or cross-sectional area on an ultrasound image (Naruse,
Trappe, and Trappe 2022). Muscle quality is assessed via echogenicity,
wherein relatively higher ultrasound image brightness indicates a
greater proportion of non-contractile tissue, such as fibrous connective
tissue in the muscle (Oranchuk et al. 2024). A related property is
intramuscular fat, which results in increased echogenicity and
observable patterns of fat infiltration in ultrasound images (Young et
al. 2015). Muscle stiffness indicates muscle tissue resistance to
deformation, measured with a high-resolution ultrasound machine capable
of strain elastography (Chino et al. 2014).

\subsection{Stepping as a Composite of Sensory Input, Central
Processing, and Motor
Output}\label{stepping-as-a-composite-of-sensory-input-central-processing-and-motor-output}

While sensory input and central processing can be operationalized and
quantified, the focus in this research is on the specific contributions
of neuromuscular activity and muscle morphology as they relate to the
task of stepping in response to a cue. Muscle structure and function
decline as part of the normal aging process, which may relate to reduced
speed and quality of the balance recovery stepping response. To better
understand this relationship, it is necessary to objectively measure
stepping reaction in a standardized manner. Lord and Fitzpatrick (2001)
developed the Choice Stepping Reaction Time task as a composite measure
of the neurological and physiological processes involved in the
initiation of a fast and appropriate stepping response. The task
requires participants to step as quickly as possible onto a designated
target to the front, side or back with the indicated foot in response to
a visual cue {[}Figure 4{]}. In the typical laboratory setup, research
participants stand on a force plate and examiners use motion capture
video to assess speed and kinematic characteristics of the stepping
response. As a composite measure of fall risk, the task is intended to
measure cognitive-motor function, coordination, balance, reaction speed,
and overall fall risk.

\subsection{Pertinent Muscles in Balance
Recovery}\label{pertinent-muscles-in-balance-recovery}

The hypothesis of this research project suggests that neuromuscular
activity and muscle morphology influence stepping performance.
Therefore, it is important to identify the muscles involved and the
muscle features most relevant to this investigation. Previous research
on this topic suggests that certain features of hip muscle strength,
activation, and structure are predictive of performance in balance and
mobility tasks (Lanza et al. 2022). Amongst older adults, capacity of
hip abductor torque (i.e., rotational force of muscles used to bring the
leg out to the side) is related to protective lateral, or side, stepping
and inversely related to falling {[}Figure 5B, 5C{]}. Additionally,
activation of the hip abductor muscles occurs later during lateral
stepping in older adults, and neuromuscular activity of hip abductors
predominates during the stance phase of walking {[}Figure 5B{]}. Older
adults with a higher propensity for falling have greater intramuscular
fat in the gluteus medius but not the tensor fascia latae. Not all
combinations of hip muscle strength, activation, and structure were
compared to all balance and mobility tasks between younger and older
adults in any one study. Furthermore, much of the previous research
systematically reviewed by Lanza et al. (2022) focused on just one or
two muscles: the gluteus medius {[}Figure 5B{]} and tensor fascia latae
{[}Figure 5C{]}. Taken together, these findings and limitations suggest
that hip muscle features may be predictive of fall risk among older
adults, but a better understanding of this relationship is needed to
inform interventions for preventing falls.

\section{Data Components}\label{data-components}

As of the time of writing this report, the data encompasses 21
participants, divided into two age-based cohorts: 13 younger adults and
8 older adults. The study aims to achieve a balanced sample size of 30
participants, with equal representation of 15 participants in each age
group. We do not yet have access to the dataset and thus do not know the
structure or specific contents of the data files.

\subsection{Measurements}\label{measurements}

For the first aim, we will analyze six dependent variables that capture
different aspects of stepping performance. These variables are
subdivided into two categories: temporal measures and speed measures.
The temporal component comprises three weight transfer time measurements
during lateral, forward, and backward stepping maneuvers. Complementing
these are three corresponding speed measurements for each stepping
direction, providing a comprehensive assessment of movement efficiency
and control.

The second aim delves deeper into the biomechanical underpinnings of
stepping performance by examining the relationship between muscle
architecture and functional movement. This analysis maintains age group
as the independent variable but expands to encompass variables that
characterize neuromuscular function and structural properties.

The dependent variables for this aim include:

\begin{itemize}
\item
  Neuromuscular activation patterns, quantified through Electromyography
  (EMG) amplitude measurements (typically measured in mA) from six key
  lower limb muscles involved in stepping.
\item
  Muscle architectural properties, assessed through four main
  parameters:
\end{itemize}

\begin{enumerate}
\def\labelenumi{\arabic{enumi}.}
\tightlist
\item
  Muscle size or volume as measure by ultrasound imaging.
\item
  Muscle quality, measured as proportion of muscle tissue to
  non-contractile connective tissue in the muscle.
\item
  Muscle stiffness, evaluated as a measure of elasticity, comparing
  muscle from unflexed to flexed (lengthened versus shortened muscle).
\item
  Intramuscular fat density, measured as proportion of fat in the
  muscle.
\end{enumerate}

To maintain analytical consistency and enable cross-comparison between
the two specific aims, we retain the same four clinical assessment
covariates from Aim 1, incorporating their respective measurements of
time, power, and force production capacity.

To account for potential confounding factors and enhance the precision
of our analysis, we expect to incorporate four standardized Clinical
Assessments as covariates:

\begin{enumerate}
\def\labelenumi{\arabic{enumi}.}
\tightlist
\item
  The Staircase Power Test (SCPT), yielding power (Joules/second)
  measurements
\item
  The Timed Up-and-Go Test, providing temporal measurements in
  milliseconds
\item
  The Four Square Step Test, also yielding temporal data in milliseconds
\item
  The Handgrip Strength Test, measuring force production in pounds
\end{enumerate}

\subsection{Description of Clinical
Assessments}\label{description-of-clinical-assessments}

The \textbf{Stair Climb Power Test (SCPT)} (Ni et al. 2017) is a
physical assessment in which participants are instructed to ascend a
flight of stairs as quickly as possible. The test begins with
participants standing at the base of the stairs, and they start climbing
upon the tester's signal of ``Ready, set, go.'' Timing starts
immediately after the ``go'' command and stops when both feet of the
participant reach the top step. Participants may use the handrail if
they deem it necessary for safety. The power output is calculated using
the formula: power = {[}(body weight in kg) x (9.8 m/s²) x (stair height
in meters){]} / (time in seconds), resulting in Joules/second. This test
is used to measure the stair climbing power by considering the
individual's body weight, the gravitational constant, the height of the
stairs, and the time taken to complete the ascent.

The \textbf{Timed Up-and-Go Test} (Bhatt et al. 2011) involves a patient
sitting in a chair with their back against the chair back. Upon the
command ``go,'' the patient stands up, walks 3 meters at a comfortable
and safe pace, turns around, walks back to the chair, and sits down.
Timing starts with the ``go'' command and ends when the patient is
seated. It is recommended to stop the timer when the patient's buttocks
contacts the chair. A practice trial is allowed but not included in the
final score. The patient must use the same assistive device each time
they are tested to ensure score comparability. This test is thus scored
by a variable of time.

In the \textbf{Four Square Step Test} (Mathurapongsakul and Siriphorn
2018), participants step over four canes arranged in a plus sign
pattern. The administrator may demonstrate the test, and participants
are allowed one practice trial to ensure understanding. Two timed trials
are conducted, with the better time recorded as the score. Participants
are instructed to complete the test as quickly as possible, ensuring
both feet touch the floor in each square while facing forward and
avoiding contact with the canes. The test begins with the participant
standing in Square 1, facing Square 2, with Square 4 positioned to the
right of Square 1. Timing starts when the participant's first foot
touches Square 2 and stops when the last foot touches Square 1. The
stepping sequence involves moving clockwise through Squares 1, 2, 3, 4,
and back to 1, then counter-clockwise through Squares 4, 3, 2, and back
to 1.

The \textbf{Handgrip Strength Test} (Abizanda et al. 2012) utilizes an
instrument that measures the force in kilograms; it is crucial for the
patient to be positioned correctly in this test. The subject should be
seated with the back, pelvis, and knees positioned as close to 90
degrees as possible. The shoulder must be adducted and neutrally
rotated, with the elbow flexed at 90 degrees, the forearm in a neutral
position, and the wrist held between 0-15 degrees of ulnar deviation.
Importantly, the arm should not be supported by either the examiner or
an armrest, and the dynamometer must be aligned vertically and in line
with the forearm. The maximum grip strength is determined by calculating
the mean of three trials.

\newpage

\section*{References}\label{references}
\addcontentsline{toc}{section}{References}

\phantomsection\label{refs}
\begin{CSLReferences}{1}{0}
\bibitem[\citeproctext]{ref-Abizanda2012}
Abizanda, P., J. L. Navarro, M. I. García-Tomás, E. López-Jiménez, E.
Martínez-Sánchez, and G. Paterna. 2012. {``Validity and Usefulness of
Hand-Held Dynamometry for Measuring Muscle Strength in
Community-Dwelling Older Persons.''} \emph{Archives of Gerontology and
Geriatrics} 54 (1): 21--27.
\url{https://doi.org/10.1016/j.archger.2011.02.006}.

\bibitem[\citeproctext]{ref-Bhatt2011}
Bhatt, T., D. Espy, F. Yang, and Y. C. Pai. 2011. {``Dynamic Gait
Stability, Clinical Correlates, and Prognosis of Falls Among
Community-Dwelling Older Adults.''} \emph{Archives of Physical Medicine
and Rehabilitation} 92 (5): 799--805.
\url{https://doi.org/10.1016/j.apmr.2010.12.032}.

\bibitem[\citeproctext]{ref-Bonsu2021}
Bonsu, A. N., S. Nousi, R. Lobo, P. H. Strutton, Q. Arshad, and A. M.
Bronstein. 2021. {``Vestibulo-Perceptual Influences Upon the
Vestibulo-Spinal Reflex.''} \emph{Experimental Brain Research} 239 (7):
2141--49. \url{https://doi.org/10.1007/s00221-021-06123-7}.

\bibitem[\citeproctext]{ref-Brooks2015}
Brooks, J. X. et al. 2015. {``Learning to Expect the Unexpected: Rapid
Updating in Primate Cerebellum During Voluntary Self-Motion.''}
\emph{Nature Neuroscience} 18: 1310--17.
\url{https://doi.org/10.1038/nn.4077}.

\bibitem[\citeproctext]{ref-CDC2024b}
Centers for Disease Control and Prevention. 2024a. {``Facts about
Falls.''} \url{https://www.cdc.gov/falls/about/index.html}.

\bibitem[\citeproctext]{ref-CDC2024a}
---------. 2024b. {``Falls Data and Statistics.''}
\url{https://www.cdc.gov/falls/data-research/facts-stats/index.html}.

\bibitem[\citeproctext]{ref-Chino2014}
Chino, K., R. Akagi, M. Dohi, and H. Takahashi. 2014. {``Measurement of
Muscle Architecture Concurrently with Muscle Hardness Using Ultrasound
Strain Elastography.''} \emph{Acta Radiologica} 55 (7): 833--39.
\url{https://doi.org/10.1177/0284185113507565}.

\bibitem[\citeproctext]{ref-Fitzpatrick1994}
Fitzpatrick, R., and D. I. McCloskey. 1994. {``Proprioceptive, Visual
and Vestibular Thresholds for the Perception of Sway During Standing in
Humans.''} \emph{Journal of Physiology} 478 (Pt 1): 173--86.
\url{https://doi.org/10.1113/jphysiol.1994.sp020240}.

\bibitem[\citeproctext]{ref-Kamen1996}
Kamen, G., and G. E. Caldwell. 1996. {``Physiology and Interpretation of
the Electromyogram.''} \emph{Journal of Clinical Neurophysiology} 13
(5): 366--84. \url{https://doi.org/10.1097/00004691-199609000-00002}.

\bibitem[\citeproctext]{ref-Kandel2021}
Kandel, E. R., J. D. Koester, S. H. Mack, and S. A. Siegelbaum. 2021.
\emph{Principles of Neural Science}. 6th ed. New York: McGraw-Hill
Education.

\bibitem[\citeproctext]{ref-Lanza2022}
Lanza, Marcel B, Breanna Arbuco, Alice S Ryan, Andrea G Shipper, Vicki L
Gray, and Odessa Addison. 2022. {``Systematic Review of the Importance
of Hip Muscle Strength, Activation, and Structure in Balance and
Mobility Tasks.''} \emph{Archives of Physical Medicine and
Rehabilitation} 103 (8): 1651--62.

\bibitem[\citeproctext]{ref-Lord2001}
Lord, Stephen R, and Richard C Fitzpatrick. 2001. {``Choice Stepping
Reaction Time: A Composite Measure of Falls Risk in Older People.''}
\emph{The Journals of Gerontology Series A: Biological Sciences and
Medical Sciences} 56 (10): M627--32.

\bibitem[\citeproctext]{ref-Marieb2020}
Marieb, Elaine N., and Patricia Brady Wilhelm. 2020. \emph{Human
Anatomy}. 9th ed. Pearson.

\bibitem[\citeproctext]{ref-Mathurapongsakul2018}
Mathurapongsakul, P., and A. Siriphorn. 2018. {``Four Square Step Test
with Foam Is More Accurate Than Those Without Foam for Discriminating
Between Older Adults with and Without Fall History.''} \emph{Journal of
Aging and Physical Activity} 26 (4): 624--28.
\url{https://doi.org/10.1123/japa.2017-0363}.

\bibitem[\citeproctext]{ref-Naruse2022}
Naruse, M., S. Trappe, and T. A. Trappe. 2022. {``Human Skeletal Muscle
Size with Ultrasound Imaging: A Comprehensive Review.''} \emph{Journal
of Applied Physiology} 132 (5): 1267--79.

\bibitem[\citeproctext]{ref-NCOA2024}
National Council on Aging. 2024. {``Get the Facts on Falls
Prevention.''}
\url{https://www.ncoa.org/article/get-the-facts-on-falls-prevention}.

\bibitem[\citeproctext]{ref-NSC2024}
National Safety Council. 2024. {``Older Adult Falls Statistics.''}
\url{https://injuryfacts.nsc.org/home-and-community/safety-topics/older-adult-falls/}.

\bibitem[\citeproctext]{ref-Ni2017}
Ni, M., L. G. Brown, D. Lawler, and J. F. Bean. 2017. {``Reliability,
Validity, and Minimal Detectable Change of Four-Step Stair Climb Power
Test in Community-Dwelling Older Adults.''} \emph{Physical Therapy} 97
(7): 767--73. \url{https://doi.org/10.1093/ptj/pzx039}.

\bibitem[\citeproctext]{ref-Oranchuk2024}
Oranchuk, Dustin J, Stephan G Bodkin, Katie L Boncella, and Michael O
Harris-Love. 2024. {``Exploring the Associations Between Skeletal Muscle
Echogenicity and Physical Function in Aging Adults: A Systematic Review
with Meta-Analyses.''} \emph{Journal of Sport and Health Science}.

\bibitem[\citeproctext]{ref-Purves2001}
Purves, D., G. J. Augustine, D. Fitzpatrick, et al., eds. 2001.
\emph{Neuroscience}. 2nd ed. Sunderland, MA: Sinauer Associates.
\url{https://www.ncbi.nlm.nih.gov/books/NBK11130/}.

\bibitem[\citeproctext]{ref-Robinson2022}
Robinson, D. A. 2022. {``Basic Framework of the Vestibulo-Ocular
Reflex.''} \emph{Progress in Brain Research} 267 (1): 131--53.
\url{https://doi.org/10.1016/bs.pbr.2021.10.006}.

\bibitem[\citeproctext]{ref-Young2015}
Young, Hui-Ju, Nathan T Jenkins, Qun Zhao, and Kevin K Mccully. 2015.
{``Measurement of Intramuscular Fat by Muscle Echo Intensity.''}
\emph{Muscle \& Nerve} 52 (6): 963--71.

\end{CSLReferences}



\end{document}
